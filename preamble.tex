% Inspired by the book Modern LaTeX and the corresponding source code 
% by Matt Kline (https://github.com/mrkline/modern-latex).

\usepackage[a5paper,
            inner=0.65in,outer=0.55in,top=0.75in,bottom=0.5in,
            footnotesep=11bp,
            footskip=3em,
            includefoot,
            ]{geometry}

\usepackage[fleqn]{amsmath}

\usepackage{fontspec}
\usepackage{anyfontsize}

\setmainfont[
    Ligatures=TeX,
    Numbers={Proportional,Lowercase},
    ]{EBGaramond-Regular}

\setmonofont[Scale=0.6]{Victor Mono}

\usepackage{microtype}
\usepackage[perpage,bottom,symbol*]{footmisc}

\deffootnote[1em]{1em}{1em}{\thefootnotemark}
\setfootnoterule{0.8\textwidth}

\DeclareTOCStyleEntry[%
    beforeskip=8bp,
    entryformat = \bfseries,
    entrynumberformat = \addfontfeature{Numbers={Tabular,Uppercase}},
    pagenumberformat = \addfontfeature{Numbers={Tabular,Uppercase}},
    linefill = \TOCLineLeaderFill
]{tocline}{chapter}
\DeclareTOCStyleEntry[%
    beforeskip=1bp,
    entrynumberformat = \addfontfeature{Numbers={Tabular,Uppercase}},
    pagenumberformat = \addfontfeature{Numbers={Tabular,Uppercase}},
    indent=0.4in 
]{tocline}{section}
\DeclareTOCStyleEntry[%
    beforeskip=0bp,
    entrynumberformat = \addfontfeature{Numbers={Tabular,Uppercase}},
    pagenumberformat = \addfontfeature{Numbers={Tabular,Uppercase}},
    indent=0.6in 
]{tocline}{subsection}

\addtokomafont{descriptionlabel}{\rmfamily}
\setkomafont{disposition}{\rmfamily}
\addtokomafont{chapter}{\addfontfeature{Numbers=Uppercase}}
\setkomafont{section}{\Large\itshape}
\setkomafont{subsection}{\large\itshape}

\setcapwidth[c]{.75\textwidth}
\setkomafont{caption}{\sffamily\footnotesize}
\setkomafont{captionlabel}{\sffamily\footnotesize}
\renewcommand*{\figureformat}{}
\renewcommand*{\tableformat}{}
\renewcommand*{\captionformat}{}

\usepackage{enumitem}
\setlist[enumerate]{font=\addfontfeatures{Numbers=Uppercase}}
\setlist[description]{leftmargin=1em}

\usepackage{tabularx}

\usepackage[draft=false]{scrlayer-scrpage}
\clearpairofpagestyles
\pagestyle{scrheadings}
\setkomafont{pagefoot}{\upshape}
\lefoot*{\thepage}
\rofoot*{\thepage}

\usepackage{changepage}
\usepackage{multicol}

\usepackage{endnotes}
\renewcommand\makeenmark{{\addfontfeature{VerticalPosition=Superior}\theenmark}}

\usepackage{scrhack}

% Python formatting
% See
% https://tex.stackexchange.com/questions/83882/how-to-highlight-python-syntax-in-latex-listings-lstinputlistings-command

% Custom colors
\usepackage{color}
\definecolor{deepblue}{rgb}{0,0,0.5}
\definecolor{deepred}{rgb}{0.6,0,0}
\definecolor{deepgreen}{rgb}{0,0.5,0}

\usepackage{listings}

%%%%%%%%%%%%%%%%%%%%%%%%%%%%%%%%%%%%%%%%%%%%%%%%%%%%%%%%%%%%%%%%%%%%%%%%%%
% Python listings

% Python style 
\newcommand\pythonstyle{\lstset{
language=Python,
basicstyle=\ttfamily\linespread{0.7}\selectfont,
keywordstyle=\ttfamily\bfseries\linespread{0.7}\selectfont\color{deepblue},
%morekeywords={self},              % Add keywords here
%emph={MyClass,__init__},          % Custom highlighting
emphstyle=\ttfamily\bfseries\color{deepred},    % Custom highlighting style
stringstyle=\color{deepgreen},
frame=tb,                         % Any extra options here
showstringspaces=false,
numbers=left,
numberstyle=\ttfamily\small
}}

% Python environment
\lstnewenvironment{python}[1][]
{
\pythonstyle
\lstset{#1}
}
{}

% Python for external files
\newcommand\pythonexternal[2][]{{
\pythonstyle
\lstinputlisting[#1]{#2}}}

% Python for inline
\newcommand\pythoninline[1]{{\pythonstyle\lstinline!#1!}}

%%%%%%%%%%%%%%%%%%%%%%%%%%%%%%%%%%%%%%%%%%%%%%%%%%%%%%%%%%%%%%%%%%%%%%%%%%
% Ini listings

\lstdefinelanguage{Ini}
{
    basicstyle=\ttfamily\linespread{0.7}\selectfont,
    columns=fullflexible,
    morecomment=[s][\color{black}\bfseries]{[}{]},
    morecomment=[l]{\#},
    morecomment=[l]{;},
    commentstyle=\color{gray}\ttfamily\linespread{0.7}\selectfont,
    morekeywords={},
    otherkeywords={=,:},
    keywordstyle={\color{green}\bfseries}
}

% Ini style 
\newcommand\inistyle{\lstset{
language=Ini,
frame=tb,                         % Any extra options here
showstringspaces=false,
numbers=left,
numberstyle=\ttfamily\small
}}

% Ini environment
\lstnewenvironment{ini}[1][]
{
\inistyle
\lstset{#1}
}
{}

% Ini for external files
\newcommand\iniexternal[2][]{{
\inistyle
\lstinputlisting[#1]{#2}}}

%%%%%%%%%%%%%%%%%%%%%%%%%%%%%%%%%%%%%%%%%%%%%%%%%%%%%%%%%%%%%%%%%%%%%%%%%%
% Figure environments 

\newenvironment{leftfigure}
    {\par\vspace{0.5\baselineskip minus 0.3\baselineskip}\begin{adjustwidth}{22bp}{0pt}}
    {\end{adjustwidth}\vspace{0.5\baselineskip minus 0.3\baselineskip}}

\newenvironment{flushleftfigure}
    {\par\vspace{0.5\baselineskip minus 0.3\baselineskip}\noindent\ignorespacesafterend}
    {\vspace{0.5\baselineskip minus 0.3\baselineskip}\par\noindent\ignorespacesafterend}

\newenvironment{centerfigure}
    {\par\vspace{0.5\baselineskip minus 0.3\baselineskip}\begin{adjustwidth}{22bp}\centering}
    {\end{adjustwidth}\vspace{0.5\baselineskip minus 0.3\baselineskip}}

\usepackage{graphicx}
\usepackage{csquotes}

\usepackage[unicode,pdfusetitle,hidelinks]{hyperref}

\newcommand{\punckern}{\kern-0.2ex}
\newcommand{\quotekern}{\kern-0.6ex}

\newcommand{\monobox}[1]{\mbox{\texttt{#1}}}
\newcommand{\introduce}[1]{\textit{#1}}
\newcommand{\acronym}[1]{\textsc{\addfontfreature{LetterSpace=5}#1}}
\newcommand{\chapref}[1]{chapter!\ref{#1}}
\newcommand{\http}[1]{\href{http://#1}{\texttt{http://#1}}}
\newcommand{\https}[1]{\href{https://#1}{\texttt{https://#1}}}
\newcommand{\edition}{First edition}

\makeatletter
\let\runauthor\@author
\let\rundate\@date
\let\runtitle\@title
\makeatother

\emergencystretch=1ex
\widowpenalty=10000
\clubpenalty=10000

